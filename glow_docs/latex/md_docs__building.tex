This document contains detailed information about building the compiler on different systems and with different options.

\subsection*{Building with J\+I\+T/\+Open\+CL Backends}

By default Glow builds with only the interpreter backend enabled. To enable support for the J\+IT and/or Open\+CL backends, pass additional options to cmake\+:


\begin{DoxyCode}
-DGLOW\_WITH\_CPU=1 -DGLOW\_WITH\_OPENCL=1
\end{DoxyCode}


\subsubsection*{Open\+CL on Ubuntu}

If you decide to use Open\+CL, the easiest way is to install portable open source implementation of the Open\+CL standard, \href{https://github.com/pocl/pocl}{\tt pocl}. Glow relies on pocl to run Open\+CL tests on CI. All required steps are outlined in the \href{https://github.com/pytorch/glow/blob/master/.circleci/build.sh#L9}{\tt install\+\_\+pocl} method.

Alternatively, you can follow these steps\+:


\begin{DoxyEnumerate}
\item Install necessary packages\+:
\end{DoxyEnumerate}


\begin{DoxyCode}
sudo apt-get install ocl-icd-opencl-dev ocl-icd-libopencl1 opencl-headers \(\backslash\)
    clinfo
\end{DoxyCode}



\begin{DoxyEnumerate}
\item Install the appropriate runtime for your C\+P\+U/\+G\+PU. This will depend on your hardware. If you have an Intel C\+PU with onboard graphics, you can navigate to Intel\textquotesingle{}s compute-\/runtime releases page on Github at \href{https://github.com/intel/compute-runtime/releases/}{\tt https\+://github.\+com/intel/compute-\/runtime/releases/} and follow their instructions. You will probably want to choose the latest release and then download and install about $\sim$4 prebuilt packages. At the time of this writing, the prebuilt packages of compute-\/runtime Release 18.\+45.\+11804 ran successfully with Glow on an Intel Core i7-\/7600U running Ubuntu 16.\+04.\+1.
\item To determine if installation was successful, you can run the following command\+:
\end{DoxyEnumerate}


\begin{DoxyCode}
clinfo
\end{DoxyCode}


This will display information about your Open\+CL platforms and devices (if found). Lastly, build Glow with the cmake flag {\ttfamily -\/\+D\+G\+L\+O\+W\+\_\+\+W\+I\+T\+H\+\_\+\+O\+P\+E\+N\+CL=ON} and run the test {\ttfamily O\+C\+L\+Test}.

\subsubsection*{Supporting multiple targets}

The J\+IT is able to target all environments supported by L\+L\+VM. If the {\ttfamily build\+\_\+llvm.\+sh} script is used to build L\+L\+VM for Glow, all the currently stable architectures will be enabled. If you wish to control which architectures are built, you can use the {\ttfamily L\+L\+V\+M\+\_\+\+T\+A\+R\+G\+E\+T\+S\+\_\+\+T\+O\+\_\+\+B\+U\+I\+LD} cmake parameter when building L\+L\+VM.

\subsection*{Building with the Sanitizers}

The clang-\/sanitizer project provides a number of libraries which can be used with compiler inserted instrumentation to find a variety of bugs at runtime. These include memory issues due such as use-\/after-\/free or double-\/free. They can also detect other types of problems like memory leaks. Glow can be built with the sanitizers enabled using an additional parameter to cmake.

The following sanitizers are currently configurable\+:


\begin{DoxyItemize}
\item Address
\item Memory
\item Undefined
\item Thread
\item Leaks
\end{DoxyItemize}

You can pass one of the above as a value to the cmake parameter {\ttfamily G\+L\+O\+W\+\_\+\+U\+S\+E\+\_\+\+S\+A\+N\+I\+T\+I\+Z\+ER}. {\ttfamily Address} and {\ttfamily Undefined} are special in that they may be enabled simultaneously by passing {\ttfamily Address;Undefined} as the value. Additionally, the {\ttfamily Memory} sanitizer can also track the origin of the memory which can be enabled by using {\ttfamily Memory\+With\+Origins} instead of {\ttfamily Memory}.

\subsection*{Building with clang-\/tidy}

The clang project provides an additional utility to scan your source code for possible issues. Enabling {\ttfamily clang-\/tidy} checks on the source code is easy and can be done by passing an additional cmake parameter during the configure step.


\begin{DoxyCode}
-DCMAKE\_CXX\_CLANG\_TIDY=$(which clang-tidy)
\end{DoxyCode}


Adding this to the configure step will automatically run clang-\/tidy on the source tree during compilation. Use the following configuration to enable auto-\/fix and to enable/disable specific checks\+:


\begin{DoxyCode}
-DCMAKE\_CXX\_CLANG\_TIDY:STRING="$(which clang-tidy);-checks=...;-fix"
\end{DoxyCode}


\subsection*{Building with coverage}

Glow uses gcov, lcov, and genhtml to generate coverage reports for the code base. Using this tool allows you to make sure that corner cases are covered with the unit tests as well as keep the unit test coverage at a healthy level. You can generate a coverage report by providing additional options to cmake\+:


\begin{DoxyCode}
-DGLOW\_USE\_COVERAGE=1
\end{DoxyCode}
 and then invoke {\ttfamily glow\+\_\+coverage} make target.

\subsection*{Building Doxygen documentation}

Building documentation can be enabled by passing an additional cmake parameter\+:


\begin{DoxyCode}
-DBUILD\_DOCS=ON
\end{DoxyCode}


The output will be placed in the {\ttfamily docs/html} subdirectory of the build output directory.

\subsection*{Generate dependency graph for cmake targets}

Run cmake normally and then execute {\ttfamily dependency\+\_\+graph} target. The {\ttfamily dependency\+\_\+graph} file contains dependencies for all project targets. It might look a bit overwhelming, in that case you could check {\ttfamily dependency\+\_\+graph.\+loader}, {\ttfamily dependency\+\_\+graph.\+cifar10}, etc. 