 \href{https://travis-ci.org/pytorch/glow}{\tt } \href{https://fb-glow-assets.s3.amazonaws.com/coverage/coverage-master/index.html}{\tt }

Glow is a machine learning compiler and execution engine for hardware accelerators. It is designed to be used as a backend for high-\/level machine learning frameworks. The compiler is designed to allow state of the art compiler optimizations and code generation of neural network graphs. This library is in active development. The project plan is described in the Github issues section and in the \href{https://github.com/pytorch/glow/wiki/Glow-Roadmap}{\tt Roadmap} wiki page.

\subsection*{Partners}

Contributions to Glow are welcomed and encouraged! Glow is developed in collaboration with the following partners\+:

\tabulinesep=1mm
\begin{longtabu} spread 0pt [c]{*{3}{|X[-1]}|}
\hline
\rowcolor{\tableheadbgcolor}\PBS\centering \textbf{  }&\PBS\centering \textbf{  }&\PBS\centering \textbf{   }\\\cline{1-3}
\endfirsthead
\hline
\endfoot
\hline
\rowcolor{\tableheadbgcolor}\PBS\centering \textbf{  }&\PBS\centering \textbf{  }&\PBS\centering \textbf{   }\\\cline{1-3}
\endhead
\PBS\centering  &\PBS\centering  &\PBS\centering  \\\cline{1-3}
\PBS\centering  &\PBS\centering  &\PBS\centering \\\cline{1-3}
\PBS\centering  &\PBS\centering  &\PBS\centering \\\cline{1-3}
\end{longtabu}


\subsection*{How does it work?}

Glow lowers a traditional neural network dataflow graph into a two-\/phase strongly-\/typed ./docs/\+IR.md \char`\"{}intermediate representation (\+I\+R)\char`\"{}. The high-\/level IR allows the optimizer to perform domain-\/specific optimizations. The lower-\/level instruction-\/based address-\/only IR allows the compiler to perform memory-\/related optimizations, such as instruction scheduling, static memory allocation and copy elimination. At the lowest level, the optimizer performs machine-\/specific code generation to take advantage of specialized hardware features. Glow features a lowering phase which enables the compiler to support a high number of input operators as well as a large number of hardware targets by eliminating the need to implement all operators on all targets. The lowering phase is designed to reduce the input space and allow new hardware backends to focus on a small number of linear algebra primitives. The design philosophy is described in an \href{https://arxiv.org/abs/1805.00907}{\tt ar\+Xiv paper}.



\subsection*{Getting Started}

\subsubsection*{System Requirements}

Glow builds and runs on mac\+OS and Linux. The software depends on a modern C++ compiler that supports C++11, on C\+Make, L\+L\+VM, glog, protocol buffers, and libpng.

\paragraph*{Get Glow!}


\begin{DoxyCode}
git clone git@github.com:pytorch/glow.git  # or: git clone https://github.com/pytorch/glow.git
cd glow
\end{DoxyCode}


\paragraph*{Submodules}

Glow depends on a few submodules\+: googletest, onnx, and a library for F\+P16 conversions.

To get them, from the glow directory, run\+:


\begin{DoxyCode}
git submodule update --init --recursive
\end{DoxyCode}


\paragraph*{mac\+OS}

Install the required dependencies using either \href{https://brew.sh/}{\tt Homebrew} or \href{https://www.macports.org/}{\tt Mac\+Ports}. If using Homebrew, run\+:


\begin{DoxyCode}
brew install cmake graphviz libpng ninja protobuf wget glog autopep8
brew install llvm@7
\end{DoxyCode}


If using Mac\+Ports, run\+:


\begin{DoxyCode}
port install cmake graphviz libpng ninja protobuf-cpp wget llvm-7.0 google-glog
\end{DoxyCode}


Note that L\+L\+VM is installed in a non-\/default location to avoid conflicts with the system\textquotesingle{}s L\+L\+VM --Homebrew usually installs L\+L\+VM in {\ttfamily /usr/local/opt/llvm/}, whereas Mac\+Ports installs it in {\ttfamily /opt/local/libexec/llvm-\/7.0/}. This means that C\+Make will need to be told where to find L\+L\+VM when building; instructions on that can be found \href{#building-with-dependencies-llvm}{\tt here}.

Finally, create a symbolic link to the Homebrew-\/ or Mac\+Ports-\/installed {\ttfamily clang-\/$\ast$} tools so that the {\ttfamily utils/format.\+sh} script is able to find them later on. For a Homebrew-\/managed installation, run\+: 
\begin{DoxyCode}
ln -s "/usr/local/opt/llvm/bin/clang-format" "/usr/local/bin/clang-format"
ln -s "/usr/local/opt/llvm/bin/clang-tidy" "/usr/local/bin/clang-tidy"
\end{DoxyCode}
 For Mac\+Ports, run\+: 
\begin{DoxyCode}
ln -s "/opt/local/libexec/llvm-7.0/bin/clang-format" "/usr/local/bin/clang-format"
ln -s "/opt/local/libexec/llvm-7.0/bin/clang-tidy" "/usr/local/bin/clang-tidy"
\end{DoxyCode}


\begin{quote}
{\bfseries Note\+:} On newer versions of mac\+OS, Xcode\textquotesingle{}s command line tools come with a non-\/traditional header layout. In order for Glow to build on newer mac\+OS versions, you might need to install {\ttfamily mac\+O\+S\+\_\+\+S\+D\+K\+\_\+headers\+\_\+for\+\_\+mac\+O\+S\+\_\+10.\+14.\+pkg} manually. For example, on Mojave this package is located in {\ttfamily /\+Library/\+Developer/\+Command\+Line\+Tools/\+Packages/}. \end{quote}


\paragraph*{Ubuntu}

\mbox{[}The following instructions have been tested on Ubuntu 16.\+04\mbox{]}

In order to build Glow on Ubuntu it is necessary to install a few packages. The following command should install the required dependencies\+:


\begin{DoxyCode}
sudo apt-get install clang clang-8 cmake graphviz libpng-dev \(\backslash\)
    libprotobuf-dev llvm-8 llvm-8-dev ninja-build protobuf-compiler wget \(\backslash\)
    opencl-headers libgoogle-glog-dev
\end{DoxyCode}
 \mbox{[}Note\+: building Glow on Ubuntu 16.\+04 with llvm-\/7 fails because llvm-\/7 xenial distribution uses an older c++ A\+BI\mbox{]}

It may be desirable to use {\ttfamily update-\/alternatives} to manage the version of clang/clang++\+:


\begin{DoxyCode}
sudo update-alternatives --install /usr/bin/clang clang \(\backslash\)
    /usr/lib/llvm-8/bin/clang 50
sudo update-alternatives --install /usr/bin/clang++ clang++ \(\backslash\)
    /usr/lib/llvm-8/bin/clang++ 50
\end{DoxyCode}


Glow uses the system default C/\+C++ compiler (/usr/bin/c++), and so you may also want to switch your default C/\+C++ compiler to clang\+:


\begin{DoxyCode}
sudo update-alternatives --config cc
    # Select the option corresponding to /usr/bin/clang ...
sudo update-alternatives --config c++
    # Select the option corresponding to /usr/bin/clang++ ...
\end{DoxyCode}


Glow {\itshape should} build just fine with gcc (e.\+g. gcc 5.\+4), but we mostly use clang and are more attentive to compatibility with clang.

Finally, in order to support the O\+N\+NX net serialization format, Glow requires {\ttfamily protobuf $>$= 2.\+6.\+1}, but the above command may install older version on older Ubuntu (e.\+g. 14.\+04). If this is the case, we suggest to look at {\ttfamily utils/install\+\_\+protobuf.\+sh} to install a newer version from source.

For details on installing Open\+CL on Ubuntu please see \href{docs/Building.md#opencl-on-ubuntu}{\tt these instructions}.

\subsubsection*{Configure and Build}

To build the compiler, create a build directory and run cmake on the source directory. It\textquotesingle{}s a good idea to build two configurations (Release and Debug) because some programs take a really long time to run in Debug mode. It\textquotesingle{}s also a good idea to build the project outside of the source directory.


\begin{DoxyCode}
mkdir build\_Debug
cd build\_Debug
cmake -G Ninja -DCMAKE\_BUILD\_TYPE=Debug ../glow
ninja all
\end{DoxyCode}


It\textquotesingle{}s possible to configure and build the compiler with any C\+Make generator, like G\+NU Makefiles, Ninja and Xcode build.

For platform-\/specific build instructions and advanced options, such as building with Address-\/\+Sanitizers refer to this guide\+: \hyperlink{md_docs__building}{Building the Compiler}.

If you\textquotesingle{}re running Mac OS v10.\+14 (Mojave) and {\ttfamily ninja all} fails because it can\textquotesingle{}t find headers (e.\+g. {\ttfamily string.\+h}), run this command to fix it, and try again. More information is available \href{https://developer.apple.com/documentation/xcode_release_notes/xcode_10_release_notes}{\tt here} under \char`\"{}\+Command Line Tools\char`\"{}.


\begin{DoxyCode}
open /Library/Developer/CommandLineTools/Packages/macOS\_SDK\_headers\_for\_macOS\_10.14.pkg
\end{DoxyCode}


\paragraph*{Building with dependencies (L\+L\+VM)}

By default, Glow will use a system provided L\+L\+VM. Note that Glow requires L\+L\+VM 7.\+0 or later. If you have L\+L\+VM installed in a non-\/default location (for example, if you installed it using Homebrew on mac\+OS), you need to tell C\+Make where to find llvm using {\ttfamily -\/\+D\+L\+L\+V\+M\+\_\+\+D\+IR}. For example, if L\+L\+VM were installed in {\ttfamily /usr/local/opt}\+:


\begin{DoxyCode}
cmake -G Ninja ../glow \(\backslash\)
    -DCMAKE\_BUILD\_TYPE=Debug \(\backslash\)
    -DLLVM\_DIR=/usr/local/opt/llvm@7/lib/cmake/llvm
\end{DoxyCode}


If L\+L\+VM is not available on your system you\textquotesingle{}ll need to build it manually. Run the script \textquotesingle{}{\ttfamily /utils/build\+\_\+llvm.sh} to clone, build and install L\+L\+VM in a local directory. You will need to configure Glow with the flag {\ttfamily -\/\+D\+L\+L\+V\+M\+\_\+\+D\+IR} to tell the build system where to find L\+L\+VM given the local directory you installed it in (e.\+g. {\ttfamily -\/\+D\+L\+L\+V\+M\+\_\+\+D\+IR=/path/to/llvm\+\_\+install/lib/cmake/llvm} if using {\ttfamily build\+\_\+llvm.\+sh}).

\subsection*{Testing and Running}

\subsubsection*{Unit tests}

The project has a few unit tests in the tests/unittests subdirectory. To run all of them, simply run {\ttfamily ninja test}.

\subsubsection*{C++ A\+PI examples}

A few test programs that use Glow\textquotesingle{}s C++ A\+PI are found under the {\ttfamily examples/} subdirectory. The {\ttfamily mnist}, {\ttfamily cifar10}, {\ttfamily fr2en} and {\ttfamily ptb} programs train and run digit recognition, image classification and language modeling benchmarks, respectively.

To run these programs, build Glow in Release mode, then run the following commands to download the cifar10, mnist and ptb databases.


\begin{DoxyCode}
python ../glow/utils/download\_datasets\_and\_models.py --all-datasets
\end{DoxyCode}


Now run the examples. Note that the databases should be in the current working directory.


\begin{DoxyCode}
./bin/mnist
./bin/cifar10
./bin/fr2en
./bin/ptb
./bin/char-rnn
\end{DoxyCode}


If everything goes well you should see\+:
\begin{DoxyItemize}
\item {\ttfamily mnist}\+: pictures from the mnist digits database
\item {\ttfamily cifar10}\+: image classifications that steadily improve
\item {\ttfamily fr2en}\+: an interactive French-\/to-\/\+English translator
\item {\ttfamily ptb}\+: decreasing perplexity on the dataset as the network trains
\item {\ttfamily char-\/rnn}\+: generates random text based on some document
\end{DoxyItemize}

Note that the default build mode is {\ttfamily Debug}, which means that the compiler itself is easy to debug because the binary contains debug info, lots of assertions, and the optimizations are disabled. It also means that the compiler and runtime are very slow, and the execution time can be hundreds of times slower than that of release builds. If you wish to benchmark the compiler, run long benchmarks, or release the product then you should compile the compiler in Release mode. Check the main C\+Make file for more details.

More details on testing and running Glow can be found in\+: \hyperlink{md_docs__testing}{Testing the Glow Compiler}.

\subsubsection*{Ahead-\/of-\/time Compilation}

Glow can be used to compile neural networks into object files containing native code. We provide resnet50 (both quantized and non-\/quantized versions) as an example of this capability in {\ttfamily examples/bundles/resnet50}. See \hyperlink{md_docs__a_o_t}{Creating}Standalone Executable Bundles" for more detail.

\subsection*{Contributing}

To get started contributing, please refer to the following guides\+:
\begin{DoxyItemize}
\item Contributing
\item \hyperlink{md_docs__coding_standards}{Coding Standards}
\item Code of Conduct
\end{DoxyItemize}

\subsubsection*{Communication}


\begin{DoxyItemize}
\item Forums\+: discuss implementations, research, etc\+: \href{https://discuss.pytorch.org/c/glow}{\tt https\+://discuss.\+pytorch.\+org/c/glow}. Make sure to label topic with the \href{https://discuss.pytorch.org/c/glow}{\tt \char`\"{}glow\char`\"{}} category.
\item Git\+Hub issues\+: bug reports, feature requests, install issues, R\+F\+Cs, thoughts, etc.
\end{DoxyItemize}

\subsection*{License}

Glow is licensed under the \mbox{[}Apache 2.\+0 License\mbox{]}(L\+I\+C\+E\+N\+SE). 